% \documentclass[aip,jcp,preprint,unsortedaddress,a4paper,onecolum]{revtex4-1}
\documentclass[aip,jcp,a4paper,reprint,onecolumn]{revtex4-1}
% \documentclass[aps,pre,twocolumn]{revtex4-1}
% \documentclass[aps,jcp,groupedaddress,twocolumn,unsortedaddress]{revtex4}

\usepackage[fleqn]{amsmath}
\usepackage{amssymb}
\usepackage[dvips]{graphicx}
\usepackage{color}
\usepackage{tabularx}
\usepackage{algorithm}
\usepackage{algorithmic}

\makeatletter
\makeatother

\newcommand{\recheck}[1]{{\color{red} #1}}
\newcommand{\redc}[1]{{\color{red} #1}}
\newcommand{\bluec}[1]{{\color{blue} #1}}
\newcommand{\vect}[1]{\textbf{\textit{#1}}}
\newcommand{\dd}[1]{\textsf{#1}}

\newcommand{\AT}{{\textrm{{AT}}}}
\newcommand{\EX}{{\textrm{EX}}}
\newcommand{\CG}{{\textrm{CG}}}
\newcommand{\HY}{{\Delta}}
\newcommand{\rdf}{{\textrm{rdf}}}



\begin{document}

\title{The reliability of AdResS in performing  Molecular Dynamics in a Grand Canonical fashion}
\author{Han Wang}
\affiliation{Institute for Mathematics, Freie Universit\"at Berlin, Germany}
\author{Carsten Hartmann}
\affiliation{Institute for Mathematics, Freie Universit\"at Berlin, Germany}
\author{Christof Sch\"utte}
\affiliation{Institute for Mathematics, Freie Universit\"at Berlin, Germany}
\author{Luigi Delle Site}
\email{luigi.dellesite@fu-berlin.de}
\affiliation{Institute for Mathematics, Freie Universit\"at Berlin, Germany}

\begin{abstract}
\end{abstract}

\maketitle



\subsection{New considerations}

For simplicity, here we consider the Brownian dynamics:
\begin{align}
  \dd d \vect X_t = \vect b(\vect X_t)\, \dd d t + \sigma \dd d\vect W_t
\end{align}
Where $\vect X_t$ is the trajectory
of the system, i.e.
$\vect X_t = \{\vect r_1(t), \vect r_2(t), \cdots, \vect r_N(t)\}$.
$\vect b(\vect X_t)$ is the force exerting on the system, namely
$\vect b(\vect X_t) = \{\vect F_1(\vect X_t), \vect F_2(\vect X_t), \cdots,
\vect F_N(\vect X_t)\}$. $\dd d\vect W_t$ is the standard Wiener process.
The corresponding Fokker-Planck equation is
\begin{align}\label{eqn:brownian-fp}
  \partial_t p(\vect x)
  - \frac{\sigma^2}{2}\,\nabla^2_{\vect x}\, p(\vect x) 
  + \nabla_{\vect x}\cdot [\vect b(\vect x) \,p(\vect x)] = 0
\end{align}
where $p(\vect x)$ is the phase space distribution. Here we assume both
the density and the RDF of the AdResS system are the same as those
of the reference atomistic system. We consider the Fokker-Planck equation
for the marginal distributions. The first two marginal distributions,
$p^{(1)}(\vect r_1)$ and $p^{(2)}(\vect r_1, \vect r_2)$, are correct
by the construction of AdResS method, 
in the sense that they are the same as the atomistic reference.
Without lose of gnerality,
integrating the l.h.s. of Eqn. \eqref{eqn:brownian-fp} 
with respect to variable $\vect r_4, \vect r_5, \cdots, \vect r_N$,
noticing the periodic boundary condition,
we have
\begin{align}\label{eqn:brownian-fp-1}
  \partial_t p^{(3)}(\vect r_1, \vect r_2, \vect r_3)
  -
  \frac{\sigma^2}{2} \,\tilde{\nabla}^2_{\vect x}\,
  p^{(3)}(\vect r_1, \vect r_2, \vect r_3)
  +
  \sum_{\alpha=1}^3\,\partial_\alpha\int \dd d\vect r_4\cdots\dd d\vect r_N\,
  \vect b_\alpha(\vect x) \, p(\vect x) = 0
\end{align}
where $\vect b_\alpha(\vect x)$
is the $\alpha$-th component of $\vect b(\vect x)$.
Here, it should be noted that the interaction in the system is
pairwise, therefore $\vect b_\alpha$ has the form:
\begin{align}
  \vect b_\alpha = \sum_{\beta\neq\alpha}\vect F(\vect r_\alpha - \vect r_\beta)
\end{align}
Calculating the 3rd term of the l.h.s. of Eqn. \eqref{eqn:brownian-fp-1},
we only consider $\alpha = 1$:
\begin{align*}
  \partial_1
  \int& \dd d\vect r_4\cdots\dd d\vect r_N\,
  \vect b_1(\vect x) \, p(\vect x) \\
  &=
  \partial_1
  \int \dd d\vect r_4\cdots\dd d\vect r_N\,
  \Big[
  \vect F(\vect r_1 - \vect r_2) +
  \vect F(\vect r_1 - \vect r_3) +
  \sum_{\beta = 4}^N\vect F(\vect r_1 - \vect r_\beta)
  \Big]
  p(\vect r_1, \vect r_2, \vect r_3, \vect r_4, \cdots, \vect r_N)\\
  & =
  \partial_1
  \bigg\{
  \Big[
  \vect F(\vect r_1 - \vect r_2) +
  \vect F(\vect r_1 - \vect r_3)
  \Big]\,
  p (\vect r_1, \vect r_2, \vect r_3)
  \bigg\}
  +
  \partial_1
  \sum_{\beta = 4}^N
  \int \dd d\vect r_\beta\,
  \vect F(\vect r_1 - \vect r_\beta)\,
  p (\vect r_1, \vect r_2, \vect r_3, \vect r_\beta)
\end{align*}



\section*{References}
\bibliography{ref}{}
\bibliographystyle{unsrt}





\end{document}