% \documentclass[aip,jcp,preprint,unsortedaddress,a4paper,onecolum]{revtex4-1}
\documentclass[aip,jcp,a4paper,reprint,onecolumn]{revtex4-1}
% \documentclass[aps,pre,twocolumn]{revtex4-1}
% \documentclass[aps,jcp,groupedaddress,twocolumn,unsortedaddress]{revtex4}

\usepackage[fleqn]{amsmath}
\usepackage{amssymb}
\usepackage[dvips]{graphicx}
\usepackage{color}
\usepackage{tabularx}
\usepackage{algorithm}
\usepackage{algorithmic}

\makeatletter
\makeatother

\newcommand{\recheck}[1]{{\color{red} #1}}
\newcommand{\redc}[1]{{\color{red} #1}}
\newcommand{\bluec}[1]{{\color{blue} #1}}
\newcommand{\vect}[1]{\textbf{\textit{#1}}}
\newcommand{\dd}[1]{\textsf{#1}}

\newcommand{\AT}{{\textrm{{AT}}}}
\newcommand{\EX}{{\textrm{EX}}}
\newcommand{\CG}{{\textrm{CG}}}
\newcommand{\HY}{{\Delta}}
\newcommand{\rdf}{{\textrm{rdf}}}



\begin{document}

\title{The Reliability of the Adaptive Resolution Technique in Performing  Grand Canonical-Like Molecular Dynamics Simulations}
\author{Han Wang}
\affiliation{Institute for Mathematics, Freie Universit\"at Berlin, Germany}
\author{Carsten Hartmann}
\affiliation{Institute for Mathematics, Freie Universit\"at Berlin, Germany}
\author{Christof Sch\"utte}
\affiliation{Institute for Mathematics, Freie Universit\"at Berlin, Germany}
\author{Luigi Delle Site}
\email{luigi.dellesite@fu-berlin.de}
\affiliation{Institute for Mathematics, Freie Universit\"at Berlin, Germany}

\begin{abstract}
\end{abstract}

\maketitle

% The COM, H-O and H-H RDFs are shown in Fig.~\ref{fig:tmp1a} and \ref{fig:tmp2a}

\begin{figure}
  \centering
  \includegraphics[]{fig/fig-rho.eps}
  \includegraphics[]{fig/fig-count.eps}
  \caption{Density profile and fluctuation (simulation not finished) of WCA AdResS.}
  \label{fig:den}
\end{figure}

\begin{figure}
  \centering
  \includegraphics[width=0.32\textwidth]{fig/fig-rdf-375-425.eps}
  \includegraphics[width=0.32\textwidth]{fig/fig-rdf-425-516.eps}
  \includegraphics[width=0.32\textwidth]{fig/fig-rdf-470-516.eps}
  \includegraphics[width=0.32\textwidth]{fig/fig-rdf-516-608.eps}
  \includegraphics[width=0.32\textwidth]{fig/fig-rdf-608-700.eps}
  \includegraphics[width=0.32\textwidth]{fig/fig-rdf-700-750.eps}
  \caption{Local COM $g(r)$'s.
    The red line is the curve corresponding to the reference explicit (all atom)
    simulation (AT).
    The curve obtained by employing the standard AdResS and WCA AdResS
    method is represented in green and blue, respectively.
    The hybrid region is equally
    divided into three parts: HY I, HY II and HY III, the widths of
    which are roughly equal to the cut-off radius, i.e. 9 \textsf{nm}.    
    The top part of each panel shows the region where the $g(r)$ is calculated,
    and the value of weighting function $w(x)$ there.
    From left to right, up to down, the panels correspond to the AT region, 
    the HY I subregion of $\Delta$,
    the left half of HY I subregion of $\Delta$,
    the HY II subregion of $\Delta$,
    the HY III subregion of $\Delta$,
    and the CG region, respectively.
  }
  \label{fig:tmp1a}
\end{figure}


\begin{figure}
  \centering
  \includegraphics[width=0.245\textwidth]{fig/fig-rdf-hhoh-375-425.eps}
  \includegraphics[width=0.245\textwidth]{fig/fig-rdf-hhoh-425-516.eps}
  \includegraphics[width=0.245\textwidth]{fig/fig-rdf-hhoh-516-608.eps}
  \includegraphics[width=0.245\textwidth]{fig/fig-rdf-hhoh-608-700.eps}
  \caption{Local H-H and O-H $g(r)$'s.
    The red line is the curve corresponding to the reference explicit (all atom)
    simulation (AT).
    The curve obtained by employing the standard AdResS and WCA AdResS
    method is represented in green and blue, respectively.
    The hybrid region is equally
    divided into three parts: HY I, HY II and HY III, the widths of
    which are roughly equal to the cut-off radius, i.e. 9 \textsf{nm}.    
    The top part of each panel shows the region where the $g(r)$ is calculated,
    and the value of weighting function $w(x)$ there.
    From left to right, up to down, the panels correspond to the AT region, 
    the HY I subregion of $\Delta$,
    the left half of HY I subregion of $\Delta$,
    the HY II subregion of $\Delta$,
    the HY III subregion of $\Delta$,
    and the CG region, respectively.
  }
  \label{fig:tmp2a}
\end{figure}


\begin{figure}
  \centering
  \includegraphics[width=0.9\textwidth]{fig/fig-rdf3.more.eps}
  \caption{The 3-body correlation function.  The 1st row:
    atomistic.
    The 2nd to 5th row:
    AdResS with TIF-IBI correction in the AT, HY I, HY II, and HY III
    regions, respectively.
    Different columns give different distant between
    the first two molecules: from left to right $r_{12} =
    0.27\,\textsf{nm},\ 0.33\,\textsf{nm},\  
    0.80\,\textsf{nm}$.  The $x$ axis is the variable $h_1$, while $y$
    axis is the variable $h_2$ (please refer to the Appendix for the
    definition of these variables).  The magnitude that indicated by
    different colors presents the magnitude of $g^{(3)}$.
  }
  \label{fig:tmp2b}
\end{figure}




% \section*{References}
% \bibliography{ref}{}
% \bibliographystyle{unsrt}



\end{document}