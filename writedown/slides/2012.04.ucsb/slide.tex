\documentclass{beamer}


% \usepackage{CJKnumb}\usepackage{beamerthemesplit}

\mode<article>
{
  \usepackage{beamerbasearticle}
  \usepackage{fullpage}
  \usepackage{hyperref}
}

%\usepackage{beamerthemesplit} 
%\usepackage{beamerthemeshadow}  
%\usepackage[width=2cm,dark,tab]{beamerthemesidebar}


% Setup appearance:

%\usetheme{Darmstadt}
\usefonttheme[onlylarge]{structurebold}
\setbeamerfont*{frametitle}{size=\normalsize,series=\bfseries}
\setbeamertemplate{navigation symbols}{}

\renewcommand\arraystretch{1.5}

% Standard packages

\usepackage[english]{babel}
%\usepackage[latin1]{inputenc}

\usepackage{epsf}
\usepackage{amsmath,amssymb}
\usepackage{graphicx}
\usepackage{tabularx}

% \usepackage[usenames,dvipsnames]{color}
\definecolor{shadow}{gray}{0.8}
\newcommand{\redc}[1]{{\color{red} #1}}
\newcommand{\bluec}[1]{{\color{blue} #1}}
\newcommand{\shadowc}[1]{{\color{shadow} #1}}
\definecolor{myyellow}{HTML}{FFB700}
\newcommand{\yellowc}[1]{{\color{myyellow} #1}}
\newcommand{\greenc}[1]{{\color{green} #1}}
\renewcommand{\v}[1]{\textbf{\textit{#1}}}
\renewcommand{\d}[1]{\textrm{#1}}


\usepackage{amsfonts}
\newcommand{\tickYes}{\checkmark}
\usepackage{pifont}
\newcommand{\tickNo}{\hspace{1pt}\ding{55}}

% \usetheme{Boadilla}
% \usetheme{Copenhagen}
% \usetheme{Madrid}
\usetheme{Singapore}


\begin{document}
%\title{Comparative atomistic and coarse-grained study of water: Simulation details vs. Simulation feasibility}
%\title[Optimizing SPME]{Optimizing Working Parameters of the Smooth Particle Mesh Ewald Algorithm}
\title[]{Some theoretical considerations of the Adaptive Resolution Simulation (AdResS)}
%
\author{Han Wang}
\institute[FUB] {
  Institute for Mathematics, Freie Universit\"at Berlin, Germany\\
  % Institut f\"ur Mathematik, Freie Universit\"at Berlin, Germany\\
\vskip 0.4cm
Joint with: Carsten Hartmann, Christof Sch\"utte, Luigi Delle Site (FUB) }
\date[12 Apr 2012]{12 Apr 2012}
\frame{\titlepage}

\begin{frame}{AdResS, Basic concepts}
  \begin{figure}
    \centering 
    \includegraphics[width=0.8\textwidth]{fig/system/system-oldw.eps}
  \end{figure}
  \begin{equation*}
    {\v F}_{\alpha \beta}=w(x_\alpha)w(x_\beta){\v F}_{\alpha\beta}^{AT}+[1-w(x_\alpha)w(x_\beta)]{\v F}^{CG}_{\alpha\beta}
  \end{equation*}
\end{frame} 

\begin{frame}{Coupling the two resolutions}{Representability problem}
  \begin{itemize}
  \item <1->
    Representability problem of water coarse graining:\\
    \redc{Structure (compressiblity)} v.s.     \redc{Pressure} 
  \item <2->
    Pressure corrected coarse-grained model,\\
  \item <3->
    \bluec{wrong density profile in hybrid region...}
    \begin{minipage}[t]{0.45\linewidth}
      \begin{figure}
        \includegraphics[width=1.\textwidth]{fig/pcorr-rho.eps}\hfill
      \end{figure}
    \end{minipage}\\
    \footnotesize{
      S. Poblete, et. al.,  J. Chem. Phys. \textbf{132}, 114101 (2010)\\
      With courtesy of Luigi Delle Site
    }
  \end{itemize}
\end{frame}

\begin{frame}{Coupling the two resolutions}
  {Effective grand-canonical approach
    \footnote{S. Fritsch et. al. Phy. Rev. Lett. (2012) in press.}
  }
  \begin{itemize}
    \vfill
  \item <1-> Use the structurally correct coarse-grained model (fit $g(r)$): 
    \bluec{$-P_{AT}V\neq -P_{CG}V$}; \bluec{$\rho\neq \rho_{0}$}; \\
    \redc{$g_{AT}(r) =  g_{CG}(r)$}; \redc{$\kappa_{AT} =  \kappa_{CG}$}
    \vfill
  \item <2-> The ``thermodynamic force'':
    \begin{align*}
      &-\left[P_{AT}+ \frac{\rho_{0}}{M_\alpha}\int_{\Delta} \redc{{\v F}^\text{th}(x)}\,\text{d}x\right] V = -P_{CG}V; \\
      &{\v F}_{\alpha}=\sum_{\beta}{\v F}_{\alpha\beta}+{\v F}^{\textrm{th}}(x_{\alpha}).
    \end{align*}
    \vfill
  \item <3-> Is AdResS a \redc{grand canonical} simulation:
    \begin{align*}
      p(\v x, N) = \frac{1}{\mathcal Z}
      e^{\beta\mu_{AT} N - \beta \mathcal H^{AT}(\v x)} \quad (?)
    \end{align*}
    \vfill
  \end{itemize}
\end{frame}


\begin{frame}{Proof}
  \begin{itemize}
  \item <1-> Notation:\\
    atomistic $\v x_1; N_1$,
    hybrid $\v x_2; N_2$,
    coarse-grained $\v x_3; N_3$.\\
    so the target is:
    \begin{align*}
      p(\v x_1, N_1) = \frac{1}{\mathcal Z_1}
      e^{\beta\mu_{AT} N_1 - \beta \mathcal H^{AT}(\v x_1)} 
    \end{align*}
  \item <2-> Firstly, fix the number of molecules in atomistic region:
    \begin{align*}
      p(\v x_1, N_1) = p(\v x_1 | N_1) \,{p (N_1)}
    \end{align*}
    we should prove:
    \begin{align*}
      p(\v x_1 | N_1) &= \frac{1}{Z_{N_1}} e^{-\beta \mathcal H^{AT}(\v x_1)} \\
      p(N_1) &= \frac{1}{\mathcal Z_1}e^{\beta\mu_{AT} N_1}Z_{N_1}
    \end{align*}
  \end{itemize}
\end{frame}

\begin{frame}{Proof}
  \begin{itemize}
  \item <1-> The atomistic region is a sub system embedded in the
    hybrid region.
    \begin{align*}
      p(\v x_1 | N_1) = \sum_{N_2}\int
      p(\v x_1 | N_1; \v x_2, N_2) \,
      p(\v x_2, N_2 | N_1)
      \,\d d\v x_2
    \end{align*}
  \item <2-> Fortunately, we have
    \begin{align*}
      p(\v x_1 | N_1; \v x_2, N_2)
      \propto &\,
      e^{-\beta\mathcal H^{AT}(\v x_1; \v x_2, N_2)}
      \quad \redc{\textrm{AT probability!!}}\\
      \mathcal H^{{AT}}(\v x_1; \v x_2, N_2) = &\,
      \sum_{j=1}^{N_1}\frac12 m_i\v v_i^2 + 
      \sum_{i,j=1}^{N_1}\frac12 U^{{AT}}(\v r_i - \v r_j) \\
      &\,+ 
      \sum_{i=1}^{N_1}\sum_{j=N_1+1}^{N_2} U^{{AT}}(\v r_i - \v r_j) 
    \end{align*}
  \end{itemize}
\end{frame}

  % \item <3-> We denote: $p(\v x_1 | N_1) = p_{N_1}(\v x_1)$.
  %   \begin{align*}
  %     p_{N_1}(\v x_1) = \int
  %     \redc{p_{N_1}(\v x_1 | \v x_2)}
  %     \,
  %     \redc{p(\v x_2 | N_1)}
  %     \, \d d\v x_2
  %   \end{align*}

\begin{frame}{Extension of the hybrid region}
  \begin{figure}
    \centering 
    \includegraphics[width=0.8\textwidth]{fig/system/system-neww.eps}
  \end{figure}
\end{frame} 


\begin{frame}{Necessary conditions for the hybrid region}
  \begin{itemize}
    \vfill
  \item <1-> Do we have a right $p(\v x_2, N_2 | N_1)$?
    \vfill
  \item <2-> Difficult to answer: \bluec{do {not} have a Hamiltonian...}
    \vfill
  \item <3-> Necessary conditions: \redc{marginal probabilities}\\
    \vfill
    \begin{itemize}
    \item First order: \redc{$\rho(x)$}.
    \vfill
    \item Second order: \redc{$g(r)$}.
    \end{itemize}
    \vfill
  \item <4-> The thermodynamic force approach  gives the right \redc{$\rho(x)$}.
    \vfill
  \item <5-> The extension of hybrid region gives the right \redc{$g(r)$}.
    \vfill
  \end{itemize}
\end{frame}

\begin{frame}{Proof of molecule number probability}
  \begin{itemize}
    \vfill
  \item <1-> the probability of $N_1$: 
    \begin{align*}
      p(\v x_1, N_1) = p(\v x_1 | N_1) \, \redc{p (N_1)}
    \end{align*}
    \vfill
  \item <2-> Assumptions:
    \vfill
    \begin{itemize}
    \item thermodynamics limit: $N_2\ \bluec{\ll}\ N_1 \ll N_3$
      \vfill
    \item additive Hamiltonian of AT and CG regions:\\
        \vskip -.5cm
      \begin{equation*}
        \mathcal H(\v x_1, N_1; \v x_3, N_3) =
        \mathcal H^{AT}(\v x_1, N_1) + \mathcal H^{CG}(\v x_3, N_3); \qquad N_1 + N_3 = N
      \end{equation*}
    \end{itemize}
    \vfill
  \item <3-> we prove:
    \vfill
    \begin{itemize}
    \item First order accuracy: \redc{$\mu_{AT} = \mu_{CG}$}
      $\leftarrow$ thermodynamics force.
    \vfill
  \item Second order accuracy: \redc{$\kappa_{AT} = \kappa_{CG}$}
    $\leftarrow g_{AT}(r) = g_{CG}(r)$.
    \end{itemize}
    \vfill
  \end{itemize}
\end{frame}

\begin{frame}
  \vfill
  \centerline{ \Huge
    Thanks!  }
  \vfill
\end{frame}

\end{document}
