\documentclass[a4paper]{article}

\begin{document}

1. I encourage the authors to provide a better context for their
present work by addressing the literature on mixed resolution
models. In particular, I believe that Klein and coworkers (JCTC 2007),
Nielsen et al (PRL 2010), Heyden and Truhlar (JCTC 2008), and Voth and
coworkers (JPCB 2006 and JCTC 2009) – just to name a few - have all
reported various mixed resolution CG-atomistic models.

2. I encourage the authors to more carefully proofread their
manuscript, which contains a number of typos. For instance, in the Fig
3 caption ``GC'' should presumably be ``CG'' on page 10, ``momentum'' should
presumably be ``moment'' on page 13 and caption 7 “9 nm” should be
``0.9nm'', ...

3. It would be helpful if, when developing the theory, the authors
described the boundary conditions of the AdResS simulation. Is the
volume fixed? Are periodic boundary conditions employed? Is there a
global thermostat or different thermostats for each region? Many of
these details are in the appendix, but they would be helpful in the
context of introducing and describing the method.

4. The authors emphasize the importance for the CG model to reproduce
the atomistic compressibility, since this describes fluctuations in
density. The authors appear to assume that accurately reproducing the
atomistic radial distribution function (rdf) implies that the
compressibility is also accurately reproduced. It is true, in
principle, that if the CG model perfectly reproduces the atomistic
rdf, $g(r)$, for all $r$, then the CG model also reproduces the atomistic
compressibility. However, the compressibility equation is defined by
the integral of $g(r)$ – 1, which may prove to be sensitive to very
small errors in the rdf. In particular, van der Vegt and coworkers
(e.g., JCTC 2012) have demonstrated that the Kirkwood Buff integrals,
which are also defined by integrals of $g(r)$ – 1, are remarkably
sensitive to very small errors in the rdf, particularly at large $r$
which may not be so accurately determined by iterative Boltzmann
inversion. Consequently, I wonder how accurately the CG model actually
reproduces the atomistic compressibility. I would encourage the
authors to address this point. Figure 10 appears to suggest that the
compressibility is quite accurately reproduced.

5. It seems that the thermodynamic force is parameterized using an
iterative approach requiring multiple CG simulations. This point
should be more clear in the paper.

6. The thermodynamic force $F^{th}(x)$ defined in eq 4 should be
presumably applied to the mass of material at a position x. How is
this force transferred to individual particles?

7. The authors parameterize the rdf-force to reproduce $g(r)$ as a
function of $x$ in the hybrid region. How is $g(r)$ and the resulting rdf
force treated as a function of $r$ for fixed $x$? Is $r$ the distance
between particles at a given fixed x, etc?

8. The authors introduce a new weighting function $w(x)$ for the hybrid
region in eq 12. Is this applied to all AdResS forces?

9. For concreteness, it would be helpful for the reader, if the
authors would explicitly define the various distances, $d_{\textrm{AT}}$, $d_{\Delta}$
used in their water simulations, when these distances are introduced
in the theory. Without definitions, it appears that eq 12 actually
shrinks the hybrid region.

10.  In Fig 5, the authors suggest that the present case with a large
hybrid region is a “worst case scenario.” While it would seem that
this is perhaps a “worst case scenario” from the sense of
computational efficiency, it would seem to be a “best case scenario”
from the perspective of allowing a smooth gradual transition that
allows for good reintroduction of atomic detail. I would encourage the
authors to clarify/comment on this point.


11. The authors suggest that section VIII provides a rigorous proof
that the AdResS algorithm ensures that the sampling in the atomistic
region is equivalent (to second order) to the sampling for a subsystem
embedded in a larger system. However, I have several concerns about
their argument that I would encourage them to consider:

a. It is not obvious to me that eq 17 is justified and it appears to
me that Eq 17 is paramount to their conclusion, i.e., that the
atomistic subsystem samples a canonical distribution. Even if the
molecules in the hybrid region are fixed in place, it is not
immediately obvious that their effects can be modeled with atomistic
conservative pair potentials.

b. It is also not obvious to me that the effects of the CG region
(region 3) can be neglected in the proof. In particular, it seems to
me that long-ranged electrostatic interactions from molecules in
region 3 could influence the distribution in the atomistic region.

c. Finally, even if the concerns 1 and 2 above can be easily
addressed, eqs 17-19 demonstrate that, for a fixed number of molecules
in the atomistic region, those molecules sample a canonical
distribution. However, this result does not ensure that the atomistic
subsystem samples the correct grand canonical distribution, which
would also require reproducing the correct joint distribution for the
configuration and number of particles, i.e., the joint distribution
$P(x,N)$.



\end{document}
