\documentclass[12pt]{article}
\usepackage{german}
\usepackage{a4}
\usepackage{graphicx}
\usepackage{pslatex}

\pagestyle{empty}
\parindent0cm
\addtolength{\textheight}{2cm}
\renewcommand{\sfdefault}{cmss}
\renewcommand{\familydefault}{\sfdefault}

\newcommand{\kopf}{\noindent
\raisebox{2.5cm}[0cm][0cm]
{
\parbox[c]{0.88\textwidth}
{
\begin{center}
\textsc{}\\
\textsc{}
\end{center}
}
}
}



\begin{document}
\kopf



\hfill\raisebox{1.5cm}[0cm][0cm] {\parbox[t]{0.34\textwidth}
{From: Luigi Delle Site,\\
Institute for Mathematics\\
Free University of Berlin (Germany)\\
{\tt\small luigi.dellesite@fu-berlin.de}\\
}}

\noindent\raisebox{1.8cm}[0cm][0cm]
{\fbox{\parbox[t]{0.38\textwidth}
{To: Editor of Journal of Chemical Theory and Computation
}}}

\vspace{3cm}



\vspace{2cm}

\parskip 2ex
Dear Editor\\
I would like to submit the paper {\it ``Adaptive Resolution Simulation (AdResS): A smooth thermodynamic and structural transition from atomistic to coarse grained resolution and vice versa in a Grand Canonical fashion''}, by H.Wang. C.Sch\"{u}tte and myself for publication on the Journal of Chemical Theory and Computation.
The paper deals with an advancement in the methodology regarding the adaptive resolution technique in Molecular Dynamics. A recent paper has shown that the adaptive resolution idea can be put in a larger statistical mechanics concept in terms of effective Grand Canonical approach. Here we explore this connection further and show how to systematically improve the accuracy of the method. The idea is tested for the delicate case of liquid water at room conditions.\\
As experts who can judge the validity of the work we suggest the following names:\\
(1) Will Noid; Penn State\\
(2) Daan Frenkel; University of Cambridge\\
(3) Kurt Binder; University of Mainz\\
(4) Gerhard Hummer; NIDDK, National Institutes of Health\\
(5) Juan de Pablo;  University of Wisconsin-Madison 



Best Regards\\
Luigi Delle Site\\






\end{document}
