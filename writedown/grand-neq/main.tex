\documentclass[aps,pre,preprint]{revtex4-1}
% \documentclass[aps,pre,twocolumn]{revtex4-1}
% \documentclass[aps,jcp,groupedaddress,twocolumn,unsortedaddress]{revtex4}

\usepackage{amsmath}
\usepackage{amssymb}
\usepackage[dvips]{graphicx}
\usepackage{color}
\usepackage{tabularx}

\makeatletter
\makeatother

\newcommand{\recheck}[1]{{\color{red} #1}}
\newcommand{\redc}[1]{{\color{red} #1}}
\newcommand{\bluec}[1]{{\color{blue} #1}}
\renewcommand{\v}[1]{\textbf{\textit{#1}}}
\renewcommand{\d}[1]{\textsf{#1}}


\begin{document}

% \title{The Error Estimate of Force Calculation in the Inhomogeneous Molecular Systems}
% \author{Han Wang}
% \affiliation{LMAM and School of Mathematical
%   Sciences, Peking University}
% \author{Pingwen Zhang}
% % \email{pzhang@pku.edu.cn}
% \affiliation{LMAM and School of Mathematical
%   Sciences, Peking University}

% \begin{abstract}
% \end{abstract}

% \maketitle


Macroscopic properties:
\begin{align}\nonumber
  O(t) &= \langle\,\langle O_N(t) \rangle\,\rangle_N\\\label{eqn:tmp1}
  & = \lim_{T\rightarrow\infty}\frac1T\int_0^T\d d\tau\hat O\{\v r(t), \v p(t), N\}\\\label{eqn:tmp2}
  & = \sum_{N=0}^\infty\int\d d\v r\,\d d\v p\,\rho_N(\v r, \v p, t)\,\hat O_N(\v r, \v p)
\end{align}
where $\rho_N = e^{\mu N}e^{-\beta H_N}$ and $\hat O_{\bar N}(\v r,\v
p) = \hat O\{\v r(t), \v p(t), \bar N\}$. \bluec{in our case, the
  perturbed system is not equilibrium, so assuming the distribution is
  grand-canonical seems not right.}

Liouville equation:
\begin{align}\nonumber
  \frac{\partial\rho_N}{\partial t}
  & = iL_N\rho_N \\\nonumber
  & = iL_O^N\rho_N + iL_P^N\rho_N \\
  & = \{H_O^N, \rho_N\} + \{H_P^N, \rho_N\}
\end{align}
(\bluec{what are the subscripts $_O$ and $_P$ mean?}) with
\begin{align}
  \rho_N(\v r, \v p, t) = S^{\dagger}(t)\rho_N(\v r, \v p, t=0)
\end{align}
An observable $O$ of the system (... \bluec{I cannot read...})
\begin{align}\nonumber
  O_N(t)
  &= O(\v r(t), \v p(t), N) \\
  &= S_N(t) O_N(0)
\end{align}


The NE average of a given observable
\begin{align}
  \frac{\d d\hat J_N}{\d dt} = i L_N\hat J_N,
  \quad \hat J_N(\epsilon) = S_N(\epsilon)\hat J_N
\end{align}
can be obtained via an ``equilibrium average'' as (Onsager-Kubo equation)
\begin{align}\nonumber
  J_N(t)
  & = \langle J\rangle \\\nonumber
  & = \int\d d \Gamma_N \hat J_N\rho_N(t)\\\nonumber
  & = (\hat J_N, S_N^\dagger(t)\rho_0^N) \\\nonumber
  & = (S_N(t)\hat J_N, \rho_0^N)
\end{align}
and
\begin{align}
  J(t) = \sum_N J_N(t)
\end{align}
\bluec{it seems should be $J(t) = \sum_Np(N)J_N(t)$.} beyond the \bluec{...}, is this
\bluec{...}? what are the {...}.

First problem to solve, start from one equilibrium distribution with let's say $N_1$,
then (formally) I must follow $S_N(t)\hat J_N$, but during the simulation, I ... follow
a $J(N(t))$ because $N$ ... fluctuates.
what I .. think is that my simulation is long enough so that I have
\bluec{a set of inital state near enough have particle number $N_1$,
  $N_2$, $N_3$ ... etc, which is a good sampling of the
  grand-canonical ensemble. From these initial states, a bundle of
  trajectories are simulated. At a certain time $t$, we can find at
  least one of the trajectories contains $N_1$ particles }.  then ...
\begin{align}
  \langle O(t)\rangle
  = \langle\,\langle O_N(t)\rangle\,\rangle_N
  = \sum_N\int\d d\v r\d d\v p \,O_N(t)\, \rho_N(\v r, \v p, t=0)
\end{align}


% \newpage

% \bibliography{ref}{}
% \bibliographystyle{unsrt}


\end{document}
