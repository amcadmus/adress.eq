% \documentclass[aip,jcp,preprint,unsortedaddress,a4paper,onecolum]{revtex4-1}
\documentclass[aip,jcp,a4paper,reprint,onecolumn]{revtex4-1}
% \documentclass[aps,pre,twocolumn]{revtex4-1}
% \documentclass[aps,jcp,groupedaddress,twocolumn,unsortedaddress]{revtex4}

\usepackage[fleqn]{amsmath}
\usepackage{amssymb}
\usepackage[dvips]{graphicx}
\usepackage{color}
\usepackage{tabularx}
\usepackage{algorithm}
\usepackage{algorithmic}

\makeatletter
\makeatother

\newcommand{\recheck}[1]{{\color{red} #1}}
\newcommand{\redc}[1]{{\color{red} #1}}
\newcommand{\bluec}[1]{{\color{blue} #1}}
\newcommand{\greenc}[1]{{\color{green} #1}}
\newcommand{\vect}[1]{\textbf{\textit{#1}}}
\newcommand{\dd}[0]{\textsf{d}}

\newcommand{\AT}{{\textrm{{AT}}}}
\newcommand{\EX}{{\textrm{EX}}}
\newcommand{\CG}{{\textrm{CG}}}
\newcommand{\HY}{{\Delta}}
\newcommand{\rdf}{{\textrm{rdf}}}
\newcommand{\mh}{\mathcal H}
\newcommand{\kin}{\textrm{kin}}


\begin{document}

\title{Analytical investigation of the chemical potential in three-point-charge water system}
\author{Han Wang}
\author{Christof Sch\"utte}
\author{Luigi Delle Site}
\affiliation{Institute for Mathematics, Freie Universit\"at Berlin, Germany}

\begin{abstract}
\end{abstract}

\maketitle

\section{Single atom molecular system}

The Hamiltonian of this system is assumed to be
\begin{align}
  \mh = \sum_{i=1}^N\frac{\vect p_i^2}{2m_i} + \sum_{i<j}U(\vect r_{ij})
\end{align}
The partition function is therefore
\begin{align}\nonumber
  Z(N,V,T)
  &=
  \frac{1}{N! h^{3N}} \int\dd\vect p^{N}\dd\vect r^N e^{-\beta\mh}\\\nonumber
  &=
  \frac{1}{N! h^{3N}} \int\dd\vect p^{N} e^{-\beta \sum_{i=1}^N\frac{\vect p_i^2}{2m_i}}
  \int\dd\vect r^Ne^{ -\beta\sum_{i<j} U(\vect r_{ij})}
\end{align}
If we assume the same mass of all particles, then
\begin{align}
  Z(N,V,T)
  &=
  \frac{(2\pi mk_BT)^{\frac{3N}2}}{N! h^{3N}}
  \int\dd\vect r^Ne^{ -\beta\sum_{i<j} U(\vect r_{ij})}
\end{align}
According to the definition of the chemical potential:
\begin{align}\nonumber
  \mu &= \frac{\partial \{-k_BT\ln Z(N,V,T)\}}{\partial N} \\\nonumber
  &\approx
  -k_BT \ln
  \bigg[
  \frac{Z(N+1,V,T)}{Z(N,V,T)}
  \bigg] \\ \nonumber
  & =
  -k_BT \ln
  \bigg[
  \Big(
  \frac{2\pi mk_BT}{h^2}
  \Big)^{\frac32}
  \frac{V}{N+1}
  \int\dd\vect r_{N+1} \frac 1V
  \dd\vect r^N
  \frac{e^{ -\beta\sum_{1\leq i<j=\leq N} U(\vect r_{ij})}}
  {\int\dd\vect r^Ne^{ -\beta\sum_{1\leq i<j\leq N} U(\vect r_{ij})}}
  e^{-\beta \sum_{i=1}^N U(\vect r_{(N+1),i})}
  \bigg]\\
  &=
  -k_BT \ln
  \bigg[
  \Big(
  \frac{2\pi mk_BT}{h^2}
  \Big)^{\frac32}
  \frac{V}{N+1}
  \bigg]
  -k_BT \ln
  \bigg[
  \Big\langle
  e^{-\beta\Delta U}
  \Big\rangle_{\vect r_1,\cdots,\vect r^N}^{\vect r_{N+1}}
  \bigg]
\end{align}
The first term is calculated analytically. The second term is calculated by, for example,
Widom particle insertion method. The ensemble average is taken in the sense that
$\{\vect r_1, \cdots, \vect r_N\}$ are subject to the canonical distribution,
and $\vect r_{N+1}$ is uniformly distributed in volume $V$.

\section{Three-point-charge rigid water model}

Notice that for each water molecule, it has 6 degrees of freedoms (DOFs):
3 translational DOFs and 3 rotational DOFs. The partitional function is 
\begin{align} \nonumber
  Z(N,V,T)
  =\,&
  \frac{1}{N! h^{9N}}
  \int\dd\vect r^{3N}
  \prod_{i=1}^N\prod_{\alpha=1}^3
  \delta( D_{\alpha}(\vect r_{3i+1},\vect r_{3i+2}, \vect r_{3i+3}))\,
  e^{ -\beta\sum_{i<j} U(\vect r_{ij})}\\\nonumber
  &\times
  \int\dd\vect p^{3N}
  \prod_{i=1}^N\prod_{\alpha=1}^3
  \delta( C_{\alpha}(\vect p_{3i+1},\vect p_{3i+2}, \vect p_{3i+3}; \vect r_{3i+1},\vect r_{3i+2}, \vect r_{3i+3}))\,
  e^{-\beta \sum_{i=1}^{3N}\frac{\vect p_i^2}{2m_i}}\\\nonumber
  =\,&
  \frac{1}{N! h^{9N}}
  \int\dd\vect r^{3N}
  \prod_{i=1}^N\prod_{\alpha=1}^3
  \delta( D_{\alpha}(\vect r_{3i+1},\vect r_{3i+2}, \vect r_{3i+3}))\,
  e^{ -\beta\sum_{i<j} U(\vect r_{ij})}\\
  &\times
  \prod_{i=1}^N
  \int\dd\vect p_{3i+1}\dd\vect p_{3i+2}\dd\vect p_{3i+3}
  \prod_{\alpha=1}^3
  \delta( C_{\alpha}(\vect p_{3i+1},\vect p_{3i+2}, \vect p_{3i+3}; \vect r_{3i+1},\vect r_{3i+2}, \vect r_{3i+3}))\,
  e^{-\beta (\frac{\vect p_{3i+1}^2}{2m_{3i+1}} + \frac{\vect p_{3i+2}^2}{2m_{3i+2}} +
    \frac{\vect p_{3i+3}^2}{2m_{3i+3}} )}
\end{align}
where
\begin{align}\label{eqn:c-r}
  &D_{\alpha}(\vect r_{3i+1},\vect r_{3i+2}, \vect r_{3i+3}) = 0\\\label{eqn:c-p}
  \textsf{and}\quad &C_{\alpha}(\vect p_{3i+1},\vect p_{3i+2}, \vect p_{3i+3}; \vect r_{3i+1},\vect r_{3i+2}, \vect r_{3i+3}) = 0
\end{align}
are constraints according to the rigid H-O bonds and H-O-H angle. For each water molecule,
there are in total 6 constrains: 3 for positions and 3 for momenta.
Notice that the constraints for the momenta~\label{eqn:c-p} also depends
on the position: the momenta are realated to each other
in a way that depends on the orientation of the water molecule.
Since the momenta space is isotropic, we can rotate it according to the
orientation of the water, so the integrals over the momenta space have the same
form. Threrefore, we can assume that \eqref{eqn:c-p} is not explicitly
depends on the position of the atoms, or is only a function of an arbitrary
configuration of water, namely:
\begin{align}
  C_{\alpha}(\vect p_{3i+1},\vect p_{3i+2}, \vect p_{3i+3}; \vect r_{3i+1},\vect r_{3i+2}, \vect r_{3i+3}) = C_{\alpha}(\vect p_{3i+1},\vect p_{3i+2}, \vect p_{3i+3}) = 0
\end{align}

\begin{figure}
  \centering
  \includegraphics[width=0.2\textwidth]{fig/water.eps}
  \caption{A schematic plot of three-point-charge rigid water molecule.
  Its orientation is defined by vectors $\vect s_1$ and $\vect s_2$.}
  \label{fig:tmp1}
\end{figure}

We now calculate the single molecular partition funtion in the momenta space:
\begin{align}
  q(V,T) =
  \int \dd\vect p_1\dd\vect p_2\dd\vect p_3
  \prod_{\alpha=1}^3
  \delta (C_\alpha (\vect p_1,\vect p_2,\vect p_3))
  e^{-\beta (\frac{\vect p_1^2}{2 m_1} + \frac{\vect p_2^2}{2 m_2} + \frac{\vect p_3^2}{2 m_3})}
\end{align}
without lost of generality, we assume 1 and 2 are hydrogen atoms, and 3 is an oxygen atoms.
By defining ${\vect p}_1 = \vect p_1 - \vect p_3$
and ${\vect p}_2 = \vect p_2 - \vect p_3$, we have
\begin{align}\label{eqn:tmp9}
  q(V,T) =
  \int\dd\vect p_3\,
  e^{-\beta\frac{\vect p_3^2}{2m_3}}
  \int\dd{\vect p}_1\dd\vect p_2
  \prod_{\alpha=1}^3
  \delta (C_\alpha (\vect p_1-\vect p_3,\vect p_2-\vect p_3,\vect p_3))\,
  e^{-\beta \big[\frac{(\vect p_1 - \vect p_3)^2}{2 m_1} + \frac{(\vect p_2 - \vect p_3)^2}{2 m_2}\big]}
\end{align}
We assume the configuration of the water is given by two unit vectors $\vect s_1$ and
$\vect s_2$, see Fig~\ref{fig:tmp1}. Note that they are no
now, we can explicitly write down the constraints for the momenta:
\begin{align}
  \delta(C_1(\vect p_1-\vect p_3,\vect p_2-\vect p_3,\vect p_3))
  &= \delta((\vect p_1-\vect p_3) \cdot \vect s_1)   \\
  \delta(C_2(\vect p_2-\vect p_3,\vect p_2-\vect p_3,\vect p_3))
  &= \delta((\vect p_2-\vect p_3) \cdot \vect s_2)   \\
  \delta(C_3(\vect p_2-\vect p_3,\vect p_2-\vect p_3,\vect p_3))
  &= \delta (
  [(\vect p_1 - \vect p_3) - (\vect p_2 - \vect p_3)] \cdot (\vect s_1  - \vect s_2)
  ) 
\end{align}
It is obvious that replacing $\vect p_1 - \vect p_3$ by $\vect p_1$,
and $\vect p_2 - \vect p_3$ by $\vect p_2$ helps simplifing the Eq.~\eqref{eqn:tmp9}:
\begin{align}
  q(V,T) =
  \int\dd\vect p_3\,
  e^{-\beta\frac{\vect p_3^2}{2m_3}}
  \int\dd{\vect p}_1\dd\vect p_2
  \prod_{\alpha=1}^3
  \delta (C_\alpha (\vect p_1,\vect p_2; \vect s_1, \vect s_2))
  e^{-\beta \big[\frac{\vect p_1^2}{2 m_1} + \frac{\vect p_2^2}{2 m_2}\big]}
\end{align}
We denote the unit vector vertical to the molecular plain by $\vect c$. we have
\begin{align}
  \vect c = \frac{\vect s_2\times\vect s_1}{\vert \vect s_2\times\vect s_1\vert}
\end{align}
So $\{\vect s_1, \vect c, \vect s_1\times\vect c\}$ defines a local coordinate
at atom 1. So does $\{\vect s_2, \vect c, \vect s_2\times\vect c\}$ define
a local coordinate at atom 2. Transform $\vect p_1$ and $\vect p_2$ by
\begin{align}
  \vect p_1
  &= 
  d_1\vect s_1 + w_1\vect c + v_1 \vect s_1\times\vect c \\
  \vect p_2
  &=
  d_2\vect s_2 + w_2\vect c + v_2 \vect s_2\times\vect c 
\end{align}
Notice that these transforms only rotate the coordinate, so the Jacobians are 1.
The constraints are now:
\begin{align}
  &\delta(\vect p_1\cdot\vect s_1) = \delta (d_1) \\
  &\delta(\vect p_2\cdot\vect s_2) = \delta (d_2) \\
  &\delta((\vect p_1 - \vect p_2)\cdot(\vect s_1 - \vect s_2))
  = \delta (v_1(\vect s_1\times\vect c)\cdot(\vect s_1-\vect s_2) - 
  v_2(\vect s_2\times\vect c)\cdot(\vect s_1-\vect s_2))
  = \delta (\nu (v_1 - v_2) )
\end{align}
where
\begin{align}
\nu = (\vect s_1\times\vect c)\cdot(\vect s_1-\vect s_2)
= (\vect s_2\times\vect c)\cdot(\vect s_1-\vect s_2).  
\end{align}
Actually, $\vert\nu\vert = \sin(\phi_{HOH})$, where $\phi_{HOH}$ is the
H-O-H angle.
For the SPC/E water model, $\vert\nu\vert = \sin(109.47^{\circ}) = 0.94282$.
Threrefore, we have 
\begin{align}\nonumber
  q(V,T)
  &=
  \int\dd\vect p_3\,
  e^{-\beta\frac{\vect p_3^2}{2m_3}}
  h^{\frac32}
  \frac1{\vert\nu\vert}
  \int\dd v\, \dd w_1\,\dd w_2\,
  e^{-\beta
    \big[
    \frac{w_1^2 + v^2}{2 m_1} + 
    \frac{w_2^2 + v^2}{2 m_2}
    \big]
  } \\
  &=
  \frac{h^{\frac32}}{\vert\nu\vert}
  \sqrt{(2\pi m_3 k_BT)^3}
  \sqrt{2\pi m_1 k_BT}
  \sqrt{2\pi m_2 k_BT}
  \sqrt{2\pi (m_1 + m_2) k_BT}
  \\
  &=
  \frac{h^{\frac32}}{\vert\nu\vert}
  \sqrt{m_1 m_2 (m_1+m_2) m_3^3 (2\pi k_BT)^6}
\end{align}
  % \sqrt{2\pi m_1 k_BT}
  % \sqrt{2\pi m_2 k_BT}
  % \sqrt{2\pi (m_1 + m_2) k_BT}
The chemical potential is
\begin{align}\nonumber
  \mu 
  \approx\,&
  -k_BT \ln
  \bigg[
  \frac{Z(N+1,V,T)}{Z(N,V,T)}
  \bigg] \\ \nonumber
  =\,&
  -k_BT \ln
  \bigg[
  \frac{1}{N h^9}
  \frac{h^{\frac32}}{\vert \nu\vert}
  \sqrt{m_1 m_2 (m_1+m_2) m_3^3 (2\pi k_BT)^6}\\\label{eqn:mu-water-tmp1}
  &\times
  \int\dd\vect r_{3N+1}\,\dd\vect r_{3N+2}\,\dd\vect r_{3N+3}
  \prod_{\alpha=1}^3
  \delta(D_\alpha(\vect r_{3N+1}\,\vect r_{3N+2}\,\vect r_{3N+3}))
  \Big\langle e^{-\beta\Delta U} \Big\rangle_{\vect r_1, \cdots,\vect r_{3N}}
  \bigg]
  % \\ \nonumber  
  % =\,&
  % -k_BT \ln
  % \bigg[
  % \frac{V^3}{N h^6 \vert \nu\vert}
  % \sqrt{m_1 m_2 (m_1+m_2) m_3^3 (2\pi k_BT)^6}\\\nonumber
  % &\times
  % \int\dd\vect r_{3N+1}\,\dd\vect r_{3N+2}\,\dd\vect r_{3N+3}
  % \frac{1}{V^3}
  % \prod_{\alpha=1}^3
  % \delta(D_\alpha(\vect r_{3N+1}\,\vect r_{3N+2}\,\vect r_{3N+3}))
  % \Big\langle e^{-\beta\Delta U} \Big\rangle_{\vect r_1, \cdots,\vect r_{3N}}
  % \bigg] \\ \nonumber  
\end{align}
Now, we want to simplify the configurational partitioan function, i.e.
the integral in the Eq.~\eqref{eqn:mu-water-tmp1}.
It is more convenient to consider the relative position:
$\vect r_{3N+1} = \vect r_{3N+1}  - \vect r_{3N+3}$ and
$\vect r_{3N+2} = \vect r_{3N+2}  - \vect r_{3N+3}$, and the form
of the integral in Eq.~\eqref{eqn:mu-water-tmp1} does not change.
We change $\vect r_{3N+1}$ and $\vect r_{3N+1}$ into the spherical coordinate, 
namely, $\vect r_{3N+1} = (r_{3N+1}, \theta_{3N+1}, \phi_{3N+1})$ and
$\vect r_{3N+2} = (r_{3N+2}, \theta_{3N+2}, \phi_{3N+2})$.
Notice the coordinate of $\vect r_{3N+2}$ is measured with respect to
$\vect r_{3N+1}$, more specifically, $\phi_{3N+2}$ is the angle between
$\vect r_{3N+1}$ and $\vect r_{3N+2}$.
We want to calculate the partition function, which is the measure of
a radomly posited and oriented water molecule.
\begin{align}\nonumber
  Z_c =\,&
  \int \dd \vect r_{3N+3}
  \int_0^\infty\dd r_{3N+1} \int_0^{2\pi}\dd \theta_{3N+1} \int_0^\pi\dd\phi_{3N+1}
  \int_0^\infty\dd r_{3N+2} \int_0^{2\pi}\dd \theta_{3N+2} \int_0^\pi\dd\phi_{3N+2}
  \\\nonumber
  &\,
  r_{3N+1}^2\sin(\phi_{3N+1})
  r_{3N+2}^2\sin(\phi_{3N+2})
  \delta(r_{3N+1} - r_{OH})
  \delta(r_{3N+2} - r_{OH})
  \frac{1}{r_{OH}}\delta(\phi_{3N+2} - \phi_{HOH})
  \\\nonumber
  =\,&
  V\cdot 4\pi r_{OH}^2 \cdot 2\pi r_{OH}\sin(\phi_{HOH}) h^{\frac32}
\end{align}
Therefore,
\begin{align}
  \mu
  \approx &\,
  -k_BT \ln
  \bigg[
  \frac{8\pi^2 r_{OH}^3 V \sin(\phi_{HOH})}{N h^6 \vert \nu\vert}
  \sqrt{m_1 m_2 (m_1+m_2) m_3^3 (2\pi k_BT)^6}
  \bigg]
  -k_BT \ln
  \bigg[
  \Big\langle
  e^{-\beta\Delta U}
  \Big\rangle_{\vect r_1, \cdots,\vect r_{3N}}^{\vect r_{3N+1},\cdots, \vect r_{3N+3}}
  \bigg]
\end{align}
where $\big\langle
e^{-\beta\Delta U}
\big\rangle_{\vect r_1, \cdots,\vect r_{3N}}^{\vect r_{3N+1},\cdots, \vect r_{3N+3}}
$ mean the ensemble average with respect to the randomly
posited and oriented $(N+1)$-th water molecule and canonically distributed
$1,2, \cdots, N$-th water molecules.
\vskip .5cm
\noindent
To calculate the kinetic part, we have:
\begin{align}
  \mu^\kin =&\,
  -k_BT \ln
  \bigg[
  \frac{8\pi^2 r_{OH}^3 N}{V}
  \Big[\frac{m_1 m_2 (m_1+m_2)}{ (m_1 + m_2 + m_3)^3}\Big]^{\frac12}
  \bigg]
  -2k_BT \ln
  \bigg[
  \frac{V}{N}
  \Big(
  \frac{2\pi (m_1+m_2+m_3)k_BT}{h^2}
  \Big)^{\frac{3}{2}}
  \bigg] \\
  =&\,
  -k_BT \ln
  \bigg[
  \frac{8\pi^2 r_{OH}^3 N}{V}
  \Big[\frac{m_1 m_2 (m_1+m_2)}{ (m_1 + m_2 + m_3)^3}\Big]^{\frac12}
  \bigg]
  + 2 \mu_{\textrm{id}}
\end{align}
where $\mu_{\textrm{id}}$ is the chemical potential for the spherical
ideal gas.
For room temperature liquid water system,
$\mu_{\textrm{id}} = -19.3\,\textsf{kJ/mol}$,
$\mu^\kin = -19.3\times 2 + 7.6 = -31.0\,\textsf{kJ/mol}$.


\end{document}





