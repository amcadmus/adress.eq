% \documentclass[aip,jcp,preprint,unsortedaddress,a4paper,onecolum]{revtex4-1}
\documentclass[aip,jcp,a4paper,reprint,onecolumn]{revtex4-1}
% \documentclass[aps,pre,twocolumn]{revtex4-1}
% \documentclass[aps,jcp,groupedaddress,twocolumn,unsortedaddress]{revtex4}

\usepackage[fleqn]{amsmath}
\usepackage{amssymb}
\usepackage[dvips]{graphicx}
\usepackage{color}
\usepackage{tabularx}
\usepackage{algorithm}
\usepackage{algorithmic}

\makeatletter
\makeatother

\newcommand{\recheck}[1]{{\color{red} #1}}
\newcommand{\redc}[1]{{\color{red} #1}}
\newcommand{\bluec}[1]{{\color{blue} #1}}
\newcommand{\greenc}[1]{{\color{green} #1}}
\newcommand{\vect}[1]{\textbf{\textit{#1}}}
\newcommand{\dd}[0]{\textsf{d}}

\newcommand{\AT}{{\textrm{{AT}}}}
\newcommand{\EX}{{\textrm{EX}}}
\newcommand{\CG}{{\textrm{CG}}}
\newcommand{\HY}{{\Delta}}
\newcommand{\rdf}{{\textrm{rdf}}}
\newcommand{\mh}{\mathcal H}
\newcommand{\kin}{\textrm{kin}}
\newcommand{\trans}{\textrm{T}}
\newcommand{\footh}{\textrm{\tiny H}}
\newcommand{\footo}{\textrm{\tiny O}}
\newcommand{\dgenq}{\dot{\underline{\vect q}}}
\newcommand{\dgenp}{\dot{\underline{\vect p}}}
\newcommand{\genq}{{\underline{\vect q}}}
\newcommand{\genp}{{\underline{\vect p}}}


\begin{document}

\title{Analytical investigation of the chemical potential in three-point-charge water system}
\author{Han Wang}
\author{Christof Sch\"utte}
\author{Luigi Delle Site}
\affiliation{Institute for Mathematics, Freie Universit\"at Berlin, Germany}

\begin{abstract}
\end{abstract}

\maketitle

\section{The Hamiltonian approach of chemical potential}
The indexes of the molecules in the system are denoted by
the Latin alphabeta, while the atoms on the molecule are denoted
by the Greek alphabeta. The coordinate of a atom that is written
as $\vect r_{i,\alpha}$ indicates that it is the $\alpha$-th atom on
the $i$-th molecule. The COM coordinate of the molecule is denoted
by $\vect R_i$. Clearly
\begin{align}
  \vect R_i = \vect R_i(\vect r_{i,1}, \cdots, \vect r_{i,A}) =
  \sum_{\alpha=1}^A\frac{m_{i,\alpha}\vect r_{i,\alpha}}{M_i}
\end{align}
We use the potential interpolation of the
system:
\begin{align}
  V =
  \sum_{i,j} w(\vect R_i) w(\vect R_j)
  \sum_{\alpha,\beta} V^\AT (\vect r_{i,\alpha} - \vect r_{j,\beta})
  +
  \sum_{i,j}[1- w(\vect R_i) w(\vect R_j)]
  V^\CG(\vect R_i - \vect R_j)
\end{align}
The computer program actually keeps track of the dynamics of all atoms
even though the molcules are in the CG region. This treatment makes the
theoretical analysis easy, and it also saves the computational cost,
because in the CG region only the COM-COM interactions are calculated.

\section{Integrating out the rotational degrees of freedom in the partition function}

Only in this section, for simplicity, we denote the three atoms on
water molecule by $\vect r_1, \vect r_2, \vect r_3$, and the COM
coordinate by $\vect R$. The water molecule, as modeled by a rigid
body, can be fully described by the COM coordinate and three Eulerian
angles: $(\phi, \theta, \psi)$. The rotational matrix is given by
\begin{align}
  A =
  \left[
  \begin{array}{rrr}
    \cos\psi \cos\phi - \cos\theta \sin\phi \sin\psi &
    \cos\psi \sin\phi + \cos\theta \cos\phi \sin\psi &
    \sin\psi \sin\theta \\
    -\sin\psi \cos\phi - \cos\theta \sin\phi \cos\psi &
    -\sin\psi \sin\phi + \cos\theta \cos\phi \cos\psi &
    \cos\psi \sin\theta \\
     \sin\theta \sin\phi &
     -\sin\theta \cos\phi &
     \cos\theta
   \end{array}
   \right]
\end{align}
Then the atom positions  $\vect r_1, \vect r_2, \vect r_3$ are given by
\begin{align}\label{eqn:trans-1}
  \vect r_1 & = \vect R + A\cdot \tilde {\vect r}_1\\\label{eqn:trans-2}
  \vect r_2 & = \vect R + A\cdot \tilde {\vect r}_2\\\label{eqn:trans-3}
  \vect r_3 & = \vect R + A\cdot \tilde {\vect r}_3
\end{align}
where
\begin{align}
  \tilde{\vect r}_1 &= [0, y_\footh, z_\footh]^\trans\\
  \tilde{\vect r}_2 &= [0, -y_\footh, z_\footh]^\trans\\
  \tilde{\vect r}_3 &= [0, 0, z_\footo]^\trans
\end{align}
the numbers $y_\footh, z_\footh, z_\footo$ are constants determined by the geometry
of water molecule.  The transformation
\eqref{eqn:trans-1}--\eqref{eqn:trans-3} yields
\begin{align}
  \vect r_1 &= \vect R
  +
  \left[
    \begin{array}{r}
      y_\footh(\cos\psi \sin\phi + \cos\theta \cos\phi \sin\psi) +
      z_\footh \sin\psi \sin\theta\\
      y_\footh (-\sin\psi \sin\phi + \cos\theta \cos\phi \cos\psi) +
      z_\footh \cos\psi \sin\theta \\
      - y_\footh \sin\theta \cos\phi + z_\footh \cos\theta
    \end{array}
  \right] \\
  \vect r_2 &= \vect R
  +
  \left[
    \begin{array}{r}
      -y_\footh(\cos\psi \sin\phi + \cos\theta \cos\phi \sin\psi) +
      z_\footh \sin\psi \sin\theta\\
      -y_\footh (-\sin\psi \sin\phi + \cos\theta \cos\phi \cos\psi) +
      z_\footh \cos\psi \sin\theta \\
      - y_\footh \sin\theta \cos\phi + z_\footh \cos\theta
    \end{array}
  \right]\\
  \vect r_3 &= \vect R
  +
  \left[
    \begin{array}{r}
      z_\footo \sin\psi \sin\theta\\
      z_\footo \cos\psi \sin\theta \\
      z_\footo \cos\theta
    \end{array}
  \right]
\end{align}
\begin{align}
  \frac
  {\partial (\vect r_1, \vect r_2, \vect r_3)}
  {\partial (\vect R, \phi, \theta, \psi)}
  =
  \left[
  \tiny
  \begin{array}{rrrrrr}
    1 & 0 & 0 &
    y_\footh(\cos\psi \cos\phi - \cos\theta \sin\phi \sin\psi) &
    - y_\footh \sin\theta \cos\phi \sin\psi +
    z_\footh \sin\psi \cos\theta &
    y_\footh(-\sin\psi \sin\phi + \cos\theta \cos\phi \cos\psi) +
    z_\footh \cos\psi \sin\theta\\
    0 & 1 & 0 &
    y_\footh (-\sin\psi \cos\phi - \cos\theta \sin\phi \cos\psi) &
    - y_\footh \sin\theta \cos\phi \cos\psi +
    z_\footh \cos\psi \cos\theta &
    y_\footh (-\cos\psi \sin\phi - \cos\theta \cos\phi \sin\psi) -
    z_\footh \sin\psi \sin\theta \\    
    0 & 0 & 1 &
    y_\footh \sin\theta \sin\phi &
    - y_\footh \cos\theta \cos\phi - z_\footh \sin\theta &
    0 \\
    1 & 0 & 0 &
    -y_\footh(\cos\psi \cos\phi - \cos\theta \sin\phi \sin\psi) &
    y_\footh \sin\theta \cos\phi \sin\psi +
    z_\footh \sin\psi \cos\theta &
    -y_\footh(-\sin\psi \sin\phi + \cos\theta \cos\phi \cos\psi) +
    z_\footh \cos\psi \sin\theta\\
    0 & 1 & 0 &
    -y_\footh (-\sin\psi \cos\phi - \cos\theta \sin\phi \cos\psi) &
    y_\footh \sin\theta \cos\phi \cos\psi +
    z_\footh \cos\psi \cos\theta &
    -y_\footh (-\cos\psi \sin\phi - \cos\theta \cos\phi \sin\psi) -
    z_\footh \sin\psi \sin\theta \\    
    0 & 0 & 1 &
    -y_\footh \sin\theta \sin\phi &
    y_\footh \cos\theta \cos\phi - z_\footh \sin\theta &
    0 \\    
    1 & 0 & 0 &
    0 &
    z_\footo \sin\psi \cos\theta &
    z_\footo \cos\psi \sin\theta \\
    0 & 1 & 0 &
    0 &
    z_\footo \cos\psi \cos\theta &
    -z_\footo \sin\psi \sin\theta \\
    0 & 0 & 1 &
    0 &
    -z_\footo \sin\theta &
    0
  \end{array}
  \right]
\end{align}

\section{Partition function for rigid water model}

We use the generalized coordinate for the system: $(\vect R, \phi,
\theta, \psi)$. For convenience, we denote $\genq = (\phi,
\theta, \psi)$, and $\dgenq = (\dot\phi,
\dot\theta, \dot\psi)$.  For the CG part, the kinetic energy
is written as
\begin{align}
  K =
  \sum_{i=1}^N\frac 12 M_i\dot{\vect R}_i^2 +
  \sum_{i=1}^N\frac 12 \boldsymbol\omega^\trans_i \vect J_i\boldsymbol\omega_i
\end{align}
and the potential energy is
\begin{align}
  V = \sum_{i,j} V^\CG(\vect R_i - \vect R_j)
\end{align}
where $\boldsymbol\omega$ is the angular velocity and $\vect J$ is the
initia of rotation. In Eulerian angle, the angular velocity can be expressed
as
\begin{align}
  \boldsymbol\omega =
  \left[
    \begin{array}{r}
      \sin\psi \sin\theta \dot\phi + \cos\psi \dot\theta\\
      \cos\psi \sin\theta \dot\phi - \sin\psi \dot\theta\\
      \cos\theta \dot\phi + \dot\psi
    \end{array}
  \right]
  =
  \left[
    \begin{array}{rrr}
      \sin\psi \sin\theta &  \cos\psi & 0 \\
      \cos\psi \sin\theta & -\sin\psi & 0 \\
      \cos\theta & 0 & 1
    \end{array}
  \right]
  \cdot
  \left[
    \begin{array}{r}
      \dot\phi\\
      \dot\theta\\
      \dot\psi
    \end{array}
  \right]
  :=
  \vect B \cdot \dgenq
\end{align}
therefore,
\begin{align}
  K = 
  \sum_{i=1}^N\frac 12 M_i\dot{\vect R}_i^2 +
  \sum_{i=1}^N\frac 12\,
  \dgenq_i^\trans \cdot\vect B_i^\trans \vect J_i \vect B_i\cdot\dgenq
\end{align}
The Lagrangian reads
\begin{align}
  \mathcal L = K - V = 
  \sum_{i=1}^N\frac 12 M_i\dot{\vect R}_i^2 +
  \sum_{i=1}^N\frac 12\,
  \dgenq_i^\trans \cdot\vect B_i^\trans \vect J_i \vect B_i\cdot\dgenq_i
  -
  \sum_{i,j} V^\CG(\vect R_i - \vect R_j)
\end{align}
The general momenta are
\begin{align}
  \vect P_i & =
  \frac{\partial \mathcal L}{\partial \dot {\vect R}_i} =
  M_i \dot {\vect R}_i \\
  \genp_i & =
  \frac{\partial \mathcal L}{\partial \dgenq_i} =
  \frac12 \vect B_i^\trans (\vect J_i + \vect J_i^\trans) \vect B_i\cdot\dgenq_i
  =
  \frac{\partial \mathcal L}{\partial \dgenq_i} =
  \vect B_i^\trans \vect J_i  \vect B_i\cdot\dgenq_i
\end{align}
The Hamiltonian is
\begin{align}\nonumber
  \mathcal H =&
  \sum_{i=1}^N \vect P_i\cdot \dot{\vect R}_i + 
  \sum_{i=1}^N \genp_i \cdot\dgenq_i - \mathcal L \\
  =&
  \sum_{i=1}^N\frac 12 M_i\dot{\vect R}_i^2 +
  \sum_{i=1}^N\frac 12\,
  \dgenq_i^\trans \cdot\vect B_i^\trans \vect J_i \vect B_i\cdot\dgenq_i
  +
  \sum_{i,j} V^\CG(\vect R_i - \vect R_j) \\
  :=&
  \tilde{ \mathcal H}
  +
  \sum_{i=1}^N\frac 12\,
  \dgenq_i^\trans \cdot\vect B_i^\trans \vect J_i \vect B_i\cdot\dgenq_i
\end{align}

The partition function of the system is
\begin{align}\nonumber
  Z(N, V, T)
  &=
  \frac{1}{N! 2^N h^{6N}}
  \int \dd\vect P^N \dd\vect R^N \dd\genp^N \dd\genq^N
  e^{-\beta \mathcal H} \\ \nonumber
  &=
  \frac{1}{N! h^{3N}}
  \int \dd\vect P^N \dd\vect R^N e^{-\beta \tilde{ \mathcal H}}
  \times
  \frac{1}{2^N h^{3N}}
  \int \dd\genp^N \dd\genq^N
  \exp 
  \Big\{
  -\beta \sum_{i=1}^N\frac 12\,
  \dgenq_i^\trans \cdot\vect B_i^\trans \vect J_i \vect B_i\cdot\dgenq_i
  \Big\}\\
  &=
  \tilde Z(N, V, T)
  \times 
  \frac{1}{2^N h^{3N}}
  \prod_{i=1}^N
  \int \dd\genp_i \dd\genq_i
  \exp 
  \Big\{
  -\beta \, \frac 12\,
  \dgenq_i^\trans \cdot\vect B_i^\trans \vect J_i \vect B_i\cdot\dgenq_i
  \Big\}
\end{align}
where $\tilde Z$ is the partition function of the system with
spherical particles.
To calculate the chemical potential, we insert a molecule into the
system and compare the partition function:
\begin{align}\nonumber
  \frac{Z(N+1, V, T)}
  {Z(N, V, T)}
  &=
  \frac{\tilde Z(N+1, V, T)}
  {\tilde Z(N, V, T)}
  \times
  \frac 1{2h^{3}}
  \int \dd\genp\, \dd\genq
  \exp 
  \Big\{
  -\beta \, \frac 12\,
  \dgenq^\trans \cdot\vect B^\trans \vect J \vect B\cdot\dgenq
  \Big\} \\
  &=
  \frac{\tilde Z(N+1, V, T)}
  {\tilde Z(N, V, T)}
  \times
  \frac 1{2h^{3}}
  \int \dd\genp\, \dd\genq
  \exp 
  \Big\{
  -\beta \, \frac 12\,
  \genp^\trans \cdot (\vect B^\trans \vect J \vect B)^{-1}\cdot\genp
  \Big\} 
\end{align}
there we assumed that $\vect B^\trans \vect J \vect B$ is
inversible. As a mater of fact, $ \det (\vect B^\trans \vect J \vect
B)= \sin^2\theta \cdot \det \vect J $, which is singular when $\theta
= M \pi, \ M\in \mathbb Z$.  Therefore, $\vect B^\trans \vect J \vect
B$ is inversible almost everywhere.
The single particle rotational partition function is
\begin{align}\nonumber
  \hat Q (T) &=
  \frac 1{2h^{3}}
  \int \dd\genp\, \dd\genq
  \exp 
  \Big\{
  -\beta \, \frac 12\,
  \genp^\trans \cdot (\vect B^\trans \vect J \vect B)^{-1}\cdot\genp
  \Big\} \\\nonumber
  &=
  \frac 1{2h^{3}}
  \int \dd\genq \int \dd\genp
  \exp 
  \Big\{
  -\beta \, \frac 12\,
  \genp^\trans \cdot (\vect B^\trans \vect J \vect B)^{-1}\cdot\genp 
  \Big\} \\\nonumber
  &=
  \frac 1{2h^{3}}
  \int \dd\genq \,
  (2\pi k_B T)^{\frac32} \, \vert \sin\theta\vert \cdot (\det \vect J)^{\frac12}\\
  \nonumber
  &=
  \frac {(2\pi k_B T)^{\frac32} (\det \vect J)^{\frac12}}{2h^{3}}
  \int_0^{2\pi}\dd\phi \int_0^\pi \sin\theta\,\dd\theta \int_0^{2\pi}\dd\psi\\
  &=
  4\pi^2\,\frac {(2\pi k_B T)^{\frac32} (\det \vect J)^{\frac12}}{h^{3}}
\end{align}
$\vect J$ is computed as:
\begin{align}
  \vect J =
  \left[
    \begin{array}{ccc}
      2m_\footh (y_\footh^2 + z_\footh^2) + m_\footo z_\footo^2 & 0 & 0\\
      0 & 2m_\footh z_\footh^2 + m_\footo z_\footo^2 & 0 \\
      0 & 0 & 2m_\footh y_\footh^2
    \end{array}
  \right]
\end{align}
Therefore,
\begin{align}
  \mu = \tilde\mu -
  k_B T\ln
  \bigg[
  8\pi^2\,\frac {(2\pi k_B T)^{\frac32} (\det \vect J)^{\frac12}}{h^{3}}
  \bigg]
\end{align}

For the explicit water, the Hamiltonian is 
\begin{align}
  \mathcal H =&
  \sum_{i=1}^N\frac 12 M_i\dot{\vect R}_i^2 +
  \sum_{i=1}^N\frac 12\,
  \dgenq_i^\trans \cdot\vect B_i^\trans \vect J_i \vect B_i\cdot\dgenq_i
  +
  \sum_{i,j} V^\AT(\vect R_i, \genq_i; \vect R_j, \genq_j) 
\end{align}
The partition function is
\begin{align}\nonumber
  Z(N, V, T)
  &=
  \frac{1}{N! 2^N h^{6N}}
  \int \dd\vect P^N \dd\vect R^N \dd\genp^N \dd\genq^N
  e^{-\beta \mathcal H}\\\nonumber
  &=
  \frac{1}{N! 2^N h^{6N}}
  \int \dd\vect R^N\dd \genq^N
  e^{-\beta\sum_{i,j} V^\AT(\vect R_i, \genq_i; \vect R_j, \genq_j) }
  \int \dd\vect P^N
  e^{-\beta \sum_{i=1}^N\frac{{\vect P}_i^2}{2M_i}}
  \int \dd\genp^N
  e^{-\beta \sum_{i=1}^N\frac12 \genp\cdot(\vect B_i^\trans\vect J_i\vect B_i)^{-1}\cdot\genp}\\
  &=
  \frac{(2\pi M k_BT)^\frac{3N}2 (2\pi \det(\vect J)^\frac13 k_BT)^\frac{3N}2}
  {N! 2^N h^{6N}}
  \int \dd\vect R^N\dd \genq^N
  \prod_{i=1}^N \sin\theta_i\,
  e^{-\beta\sum_{i,j} V^\AT(\vect R_i, \genq_i; \vect R_j, \genq_j) }  
\end{align}
to calculate the chemical potential, we compute:
\begin{align}\nonumber
  \frac{Z(N+1, V, T)}
  {Z(N, V, T)}
  =&\,
  \frac{(2\pi M k_BT)^\frac{3}2 (2\pi \det(\vect J)^\frac13 k_BT)^\frac{3}2 }{ 2 (N+1) h^6 }\times \\\nonumber
  &\,
  \int \dd R_{N+1} \int\sin\theta_{N+1}\dd\genq_{N+1}
  \int \dd\vect R^N\dd \genq^N
  e^{-\beta\Delta V}
  \frac{
    \prod_{i=1}^N \sin\theta_i\,
    e^{-\beta\sum_{i,j} V^\AT(\vect R_i, \genq_i; \vect R_j, \genq_j) }
  }
  {
    \int \dd\vect R^N\dd \genq^N
    \prod_{i=1}^N \sin\theta_i\,
    e^{-\beta\sum_{i,j} V^\AT(\vect R_i, \genq_i; \vect R_j, \genq_j) }
  } \\\nonumber
  =&\,
  \frac{(2\pi M k_BT)^\frac{3}2 (2\pi \det(\vect J)^\frac13 k_BT)^\frac{3}2 }{ 2 (N+1) h^6 } \times
   8\pi^2 V
  \int \frac1V\dd R_{N+1} \int \frac{\sin\theta_{N+1}}{8\pi^2}\dd\genq_{N+1}
  \Big\langle
  e^{-\beta\Delta V}
  \Big\rangle_{1,\cdots,N} \\
  =&\,
  \frac{4\pi^2 V (2\pi M k_BT)^\frac{3}2 (2\pi \det(\vect J)^\frac13 k_BT)^\frac{3}2 }{ (N+1) h^6 }
  \Big\langle
  e^{-\beta\Delta V}
  \Big\rangle_{1,\cdots,N}^{N+1}  
\end{align}
Therefore,
\begin{align}
  \mu = -k_BT \ln
  \bigg[
  \frac V{N+1}
  \Big(
  \frac{2\pi M k_BT}{h^2}
  \Big)^\frac32
  \bigg]
  -
  k_BT\ln
  \bigg[
  4\pi^2
  \Big(
  \frac{2\pi \det(\vect J)^\frac13 k_BT}{h^2}
  \Big)^\frac32
  \bigg]
  -
  k_BT\ln
  \bigg[
  \Big\langle
  e^{-\beta\Delta V}
  \Big\rangle_{1,\cdots,N}^{N+1}  
  \bigg]  
\end{align}
The average $\langle\cdot\rangle_{1,\cdots,N}^{N+1}$ means that the
DOFs of $1,\cdots,N$ molecules are subject to the canonical distribution,
while the $N+1$-th molecule is subject to the uniform distribution.


\end{document}





